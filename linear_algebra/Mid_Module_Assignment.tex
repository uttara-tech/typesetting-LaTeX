
\newcommand{\pd}[2]{\frac{\partial{#1}}{\partial{#2}}}
\newcommand{\pdd}[2]{\frac{\partial^2{#1}}{\partial{#2}^2}}

\documentclass[a4paper]{report}

\usepackage[tmargin=1in,bmargin=1in,lmargin=0.75in,rmargin=0.75in]{geometry}

\linespread{1.5}
\usepackage{fancyhdr}
\usepackage{lipsum}
\usepackage{mathtools}
\usepackage{tikz}
\usepackage{pgfplots}
\usepackage{amsmath}
\usepackage[most]{tcolorbox}
\begin{document}
\pagestyle{fancy}

\begin{titlepage}
    \begin{center}
        \vspace*{1cm}
            
        \Huge
        \textbf{Maths and Statistics for AI and Data Science}
            
        \vspace{0.5cm}
        \LARGE
        Practical Assessment – 1
            
        \vspace{1.5cm}
            
        \textbf{by Uttara Naidu}
            
        \vfill
            
        Submitted to            

        University of Liverpool\\
            
    \end{center}
\end{titlepage}     


\lhead[LO,LE]{\includegraphics[width=1.5in]{logo.png}}
%\fancyhead[LO,LE]{}

\section*{Question 1}

1. a) Function $f(x)=2x+3$ , $g(x)=x^3$ . Find $f{\circ}g$ and inverse of $f{\circ}g$ .  

b) Find the derivative of the functions  $f(x)=\sin x \ln \cos 2x+6{e^x}$ 

c) Find the critical points for function ,$f(x)=-x^3+6x^2+15x+4, x\in \mathbf{R}$  , and state if the critical points are minima or maxima 

\textit{Provide appropriate justification and explanation to all your answers, detailing the methods used.}

\subsection*{Solution:}

\subsubsection*{a) Find  $f{\circ}g$ and inverse of $f{\circ}g$}

\paragraph{a.1) Find $f{\circ}g$}

\subparagraph{}

The $f{\circ}g$ denotes a composite function (Hass, 2019), meaning a combination of functions $f(x)$ and $g(x)$. To evaluate a composite function, below definition is applied:

\begin{equation} 
    \color{blue} f{\circ}g (x)=f(g(x)) \tag{A}
\end{equation}

We have,

\begin{align}
\begin{split} \label{1}
 f(x) = 2x + 3   
\end{split}\\
\begin{split} \label{2}
    g(x) = x^3
\end{split}
\end{align}

Applying the definition $(A)$ to equations $(1)$ and $(2)$ , we get: 

\begin{align}
    \begin{split}
       (f{\circ}g)(x)& = f(g(x)) \\
      f(x^3) &= 2{x^3}+3 \\
    \end{split}
\end{align}


\paragraph{a.2) Inverse of $f{\circ}g$ i.e. $[{(f{\circ}g)(x)}]^{-1}$}

\subparagraph{}

The inverse of a function $f^{-1}$ , if $f(a)=b$ is defined by (Hass, 2019),

\begin{equation} 
    \color{blue} f^{-1}(b) = a \tag{B}
\end{equation}

Using equation (3) and solving for $x$ in terms of $y$

\begin{align*}
    \begin{split}
      f(x^3) &= 2{x^3}+3 \\
          y&=2{x^3}+3
    \end{split}
\end{align*}

\newpage Interchanging $y$ and $x$ to solve for $y$,

\begin{align*}
    x&= 2{y^3}+3 \\
    2{y^3}&= x-3 \\
    y^3&= \frac{x}{2}-\frac{3}{2}\\
    y&=\sqrt[3]{\frac{x-3}{2}}
\end{align*}

Rationalizing the denominator by multiplying the numerator and the denominator by ${\sqrt[3]{2^2}}$, we get,

\begin{align*}
    y&=\frac{\sqrt[3]{x-3}}{\sqrt[3]2} \\
    y&=\frac{\sqrt[3]{x-3}}{\sqrt[3]2} * \frac{{\sqrt[3]{2^2}}}{{\sqrt[3]{2^2}}} 
  =\frac{\sqrt[3]{x-3}}{\sqrt[3]2} * \frac{{\sqrt[3]4}}{{\sqrt[3]4}}\\
    y&=\frac{\sqrt[3]{4(x-3})}{2}
\end{align*}

Replacing $y$ with $f^{-1}$ as per definition (B),

\begin{align}
    f^{-1}(x)&=\frac{\sqrt[3]{4(x-3})}{2}
\end{align}


Therefore from equations (3) and (4), we have,

\begin{equation*}
    \boxed{(f{\circ}g)(x) = 2{x^3}+3}
\end{equation*}


\begin{equation*}
    \boxed{[(f{\circ}g)(x)]^{-1} = \frac{\sqrt[3]{4(x-3})}{2}}
\end{equation*}




\subsection*{b) Find the derivative of the given function }

\paragraph{}

We have the below function:

\begin{equation*}
    f(x)=\sin x \ln \cos 2x+6{e^x}
\end{equation*}

The sum rule of derivatives is given by,

\begin{equation*}
    \color{blue} \frac{dx}{dy}({u+v}) = \frac{du}{dx}+\frac{dv}{dx} \tag{C}
\end{equation*}

Applying this rule to the given function as below,

\begin{align*}
     f'(x)&=\frac{d}{dx}[\sin x \ln \cos 2x+6{e^x}] \\
     f'(x)&=\frac{d}{dx}\sin x \ln \cos 2x + \frac{d}{dx}6{e^x}
\end{align*}

For ease of computing, splitting the terms on right hand side as:

$$f'(x)=f'(l)+f'(r)$$


where,

\begin{align}
    f'(l)&=\frac{d}{dx}\sin x \ln \cos 2x \\
    f'(r)&=\frac{d}{dx}6{e^x}
\end{align}

\paragraph{b.1) Let's solve $f'(l)$ first:}

\subparagraph{}


The product rule of derivatives is given by,

\begin{equation*}
    \color{blue} \frac{d}{dx}({uv}) = u\frac{dv}{dx}+v\frac{du}{dx} \tag{D}
\end{equation*}

Applying this rule to $f'(l)$ from equation (5), we get,

\begin{align*}
    f'(l)=\sin x \frac{d}{dx}\ln \cos 2x + \ln \cos 2x \frac{d}{dx} \sin x
\end{align*}

The chain rule, in Leibniz's notation (Hass, 2019), if $y=f(u)$  and $u=g(x)$ is given by,

\begin{equation*}
    \color{blue} \frac{dy}{dx}=\frac{dy}{du}*\frac{du}{dx} \tag{E}
\end{equation*}

This rule is used to simplify the composite function $\ln \cos 2x$ as below,

\begin{align*}
    f'(l)= \sin x [ \frac{1}{\cos 2x} \frac{d}{dx}\cos 2x] + \ln \cos 2x *\cos x
\end{align*}

The cosine derivative rule and trignometric rule is given by,

\begin{align*} \tag{\color{blue} F}
    \begin{split} 
        \color{blue} \frac{d}{dx}(\cos x) = -\sin x 
    \end{split}\\
    \begin{split}
        \color{blue} \tan x = \frac{\sin x}{\cos x}
    \end{split}
\end{align*}

Applying both of these rules to $f'(l)$ as below,

\begin{align*}
    f'(l)&=\sin x [ \frac{1}{\cos 2x} -\sin 2x*2] + \cos x\ln \cos 2x \\
        &=-2\sin x [ \frac{\sin 2x}{\cos 2x}] + \cos x\ln \cos 2x 
\end{align*}

\begin{equation}
    f'(l)=\cos x\ln \cos 2x-2\sin x \tan 2x
\end{equation}

\paragraph{b.2) Now, solving $f'(r)$:}

\subparagraph{}

Referring to equation (6),

\begin{align*}
    f'(r)=\frac{d}{dx}6{e^x}
\end{align*}


The derivative constant rule is given by,

\begin{equation*}
    \color{blue} \frac{d}{dx}{cu} = c\frac{du}{dx} \tag{G}
\end{equation*}

Applying this rule to $f'(r)$, we get

\begin{align*}
    f'(r)=6 \frac{d}{dx}e^x
\end{align*}


The derivative rule of an exponential function is given by,

\begin{equation*}
    \color{blue} \frac{d}{dx}e^u = e^u \frac{du}{dx} \tag{H}
\end{equation*}

Applying this rule to $f'(r)$, we get

\begin{align*}
    f'(r)=6e^x\frac{d}{dx}x
\end{align*}

Applying the power rule, i.e.

\begin{equation*}
    \color{blue} \frac{d}{dx}{x^n} = x^{n-1} \tag{I}
\end{equation*}

The $f'(r)$ becomes,

\begin{align*}
    f'(r) &= 6e^x\times x^{1-1} \\
          &= 6e^x\times x^{0} \\
          &= 6e^x
\end{align*}

\begin{equation}
    f'(r)= 6e^x
\end{equation}


Combining  equations (7) and (8), we get the final derivative as below:


$$f'(x)=f'(l)+f'(r)$$



\begin{equation*}
    \boxed{f'(x)= \cos x\ln \cos 2x-2\sin x \tan 2x +6e^x}
\end{equation*}


\subsubsection*{c)  Find the critical points for function $f(x)=-x^3+6x^2+15x+4, x\in {R}$ and state if the critical points are minima or maxima.}


As per definition (Hass, 2019), "An interior point of the domain of a function $f$ where $f'$ is zero or undefined is a critical point of $f$." Critical points are the points on the graph where the function's concativity is amended i.e. changed from increasing to decreasing concativity or vice a versa. 

Following this, the steps we incorporate to determine the critical points of the given function $f(x)$ are: \newline 1. Determine the derivative $f'(x)$ of the given function
\newline 2. Equate $f'(x)$ to $0$ to calculate critical points

\paragraph{Step 1: Find $f'(x)$}

\subparagraph{}

We have,

$$f(x)=-x^3+6x^2+15x+4$$

$$ f'(x)=\frac{d}{dx}[-x^3+6x^2+15x+4]$$

Applying sum rule as given by (C) and derivative of constant function from (G), we get,

\begin{align*}
    f'(x)=\frac{d}{dx}(-x^3)+6\frac{d}{dx}(x^2)+15\frac{d}{dx}x+\frac{d}{dx}(4)
\end{align*}

Further, applying power rule of derivatives from (I),

\begin{align*}
    f'(x)&=3\times(-x)^{3-1}+6\times2(x^{2-1})+15\times x^{1-1}+0 \\
    f'(x)&=-3x^2+12x+15
\end{align*}


\paragraph{Step 2: Find critical points}

\subparagraph{}

As per the definition, to find critical points, we equate the derivative to $0$

$$f'(x) =0$$
$$-3x^2+12x+15 = 0$$

Dividing both sides by 3, we get,

\begin{align*}
    -x^2+4x+5&=0 \\
    (-1) (x^2-5x+x-5&=0 \\
    (-1)(x-5)(x+1)&=0
\end{align*}



Solving for $x$, by setting


$$\begin{array}{cc}
    x-5=0, & x=5  \\
     x+1=0, & x=-1
\end{array}$$


Evaluating function $f(x)$ at each of these values of $x$:

i) $x=5$

\begin{align*}
    f(5)&=-5^3+6\times 5^2+15 \times 5+4 \\
        &=-125+150+75+4 \\
        &=104
\end{align*}


ii) $x=-1$

\begin{align*}
    f(-1)&=-(-1^3)+6 \times (-1^2)+15 \times -1+4 \\
        &=1+6-15+4 \\
        &=-4    
\end{align*}


The critical points are:

$(5,104)$ and $(-1,-4)$


%\begin{tikzpicture}
%\begin{axis}[xmin=-6, xmax=6, ymin=-11, ymax=11, axis x line=middle, axis y line=middle]
%\addplot[domain=-3:3,samples=100]{exp(-x^3+6*exp(x^2)+15*x+4)};
%\addplot[mark=*] coordinates {(5,104)};
%\addplot[mark=*] coordinates {(-1,-4)};
%\end{axis}
%\end{tikzpicture}


\section*{Question 2}

2. Function $f(x,y)=x^2 y^3+y^2+2x(y+1)$  Suppose the initial values are given $x_0=1,y_0=0$ , and the learning rate is set to 0.01 

a) Perform one iteration of the gradient descent algorithm 

b) Find the Hessian matrix of  $f(x,y)$

\textit{Provide appropriate justification and explanation to all your answers, detailing the methods used.}


\subsection*{Solution:}

\subsubsection*{a)  Gradient Descent algorithm Iteration}

\paragraph{}


The mathematical formula for calculating iteration step size in Gradient Descent algorithm is (UoL,2025):

\begin{equation}
    \color{blue} \rho_{n+1} = \rho_n - \eta \nabla f(\rho_n) \tag{A}
\end{equation}



The given function is,

$$f(x,y)=x^2 y^3+y^2+2x(y+1)$$

Substituting the values $x_0=1$ and $y_0=0$ in the function, we get,

\begin{align*}
    f(x_0,y_0)&= x_0^2 y_0^3+y_0^2+2x_0(y_0+1) \\
    &= (1^2)\times 0+0+2\times(1) (0+1) \\
    &= 2
\end{align*}


Hence, the starting point of optimization is,

\begin{align}
    f(x_0,y_0) =2
\end{align}

As per formula (A), the first iteration can be calculated with below equation:

\begin{align}
    \rho_{x_1,y_1} = \rho_{x_0,y_0} - \eta \nabla f(\rho_{x_0,y_0})
\end{align}

Partial derivatives of functions with two variables (dependent or independent) are normal derivatives, with respect to only one variable at a time (Hass, 2019). 
\newline Based on this, below calculation represents the derivatives of $f(x,y)$ w.r.t. $x$ i.e. $f_x$ and w.r.t. $y$ i.e. $f_y$:

$$\nabla f =
\left [\begin{array}{cc}
    \underbrace{\pd{f}{x}}_{\text{{$f_x$}}} & \underbrace{\pd{f}{y}}_{\textbf{{$f_y$}}}
\end{array}\right] = \left [\begin{array} {cc}
    \pd{}{x}[x^2 y^3+y^2+2x(y+1)] & \pd{}{y}[x^2 y^3+y^2+2x(y+1)]
\end{array}\right] $$


Applying constant and power derivative rules (as mentioned in (G) and (I) earlier) and calculating partial derivatives of $f(x,y)$ as below:


\begin{equation} \nabla f =
\begin{bmatrix}
     2xy^3+2(y+1) & 3x^2y^2+2y+2x
\end{bmatrix}
\end{equation}


Substituting values from equations (9), (11) and learning rate $\eta$ in equation (10), we get,

\begin{align*}
\rho_{x_1,y_1} &= \rho_{x_0,y_0} - \eta \nabla f(\rho_{x_0,y_0}) \\
&=(1,0)-(0.01)\times [2 \cdot 1 \cdot 0 + 2(0+1), 3 \cdot (1^3)\cdot 0 + 2\cdot 0+2\cdot 1] \\
&=(1,0) - 0.01 (2,2) \\
&= (1,0) - (0.02, 0.02) \\
&=(0.98, -0.02)
\end{align*}

\begin{equation}
    \rho_{x_1,y_1} = (0.98, -0.02)
\end{equation}


Computing $f(0.98,-0.02)$ as below to get value of the function:

\begin{align*}
f(x_1,y_1)&= x_1^2 y_1^3+y_1^2+2x_1(y_1+1) \\
&=(0.98^2)(-0.02^3)+(-0.02^2)+2\cdot (0.98)(-0.02+1) \\
&=1.92
\end{align*}


\subsubsection*{b) Hessian Matrix calculation}

Hessian matrix is a matrix of second order partial derivativatives which is used to calculate the curvature information of a given function. (Sharma, 2022)

A Hessian matrix is given as below:

\[\ H_f=
\begin{bmatrix}
    \pd{^2f}{x_1^2} & \pd{^2f}{x_1x_2} & ... & \pdd{f}{x_n} \\
    \pd{^2f}{x_2x_1} & \pd{^2f}{x_2^2} & ... & \pd{^2f}{x_2x_n} \\
    .&.&.&. \\
    .&.&.&. \\
    \pd{^2f}{x_nx_1} & \pd{^2f}{x_nx_2} & ... & \pd{^2f}{x_n^2} \\
    
\end{bmatrix} \]

Hessian matrix is always a square matric with its dimension equal to the number of variables of a function. Therefore, for the function $f(x,y)$ with two variables, the dimension will be $2\times 2$ and can be written as:

\[\ H_{f(x,y)} =
\begin{bmatrix}
    \pd{^2f}{x^2} & \pd{^2f}{xy} \\
    \pd{^2f}{yx} & \pd{^2f}{y^2}     
\end{bmatrix} \]

Referring to equation (11), we have already obtained the First order partial derivatives of $f(x,y)$:

\begin{align*}
    \pd{f}{x} &= 2xy^3+2(y+1) \\
    \pd{f}{y} &= 3x^2y^2+2y+2x
\end{align*}


Second order partial derivatives of $f(x,y)$ are as below:

\begin{align*}
    \pd{^2f}{x^2}&=2y^3 \\
    \pd{^2f}{y^2} &= 6x^2y+2 \\
    \pd{^2f}{x\partial y} &= \pd{}{x}(\pd{f}{y}) = \pd{}{x}(3x^2y^2+2y+2x) = 6xy^2+2 \\
    \pd{^2f}{y\partial x} &= \pd{}{y}(\pd{f}{x}) = \pd{}{y}(2xy^3+2(y+1)) = 6xy^2+2
\end{align*}



Therefore, the final Hessian matrix is as given below:


\[\ \boxed{ H_{f(x,y)} =
\begin{bmatrix}
    2y^3 & 6xy^2+2 \\
    6xy^2+2 & 6x^2y+2     
\end{bmatrix}} \]

\newpage
\section*{Question 3}

3. Take this system of linear equations

\begin{align*}
    x+y-z&=-3 \\
2x+3y-8z&=-18 \\
5x+6y-10z&=-25)
\end{align*}


a) Write this system as a Matrix vector equation 

b) Calculate the determinant and thereby determine if there is a unique solution to this system of equations 

c) Write this as an augmented matrix and solve this system of equations 

\textit{Provide appropriate justification and explanation to all your answers, detailing the methods used.}

\subsection*{Solution:}

\subsubsection*{a)  Matrix vector equation}


A matrix equation has the form (Fred,2015):

\begin{equation}
    \color{blue} Ax=b \tag{A}
\end{equation}

where,
\newline $A$ is any $m\times m$  or $m\times n$ matrix called as coefficient matrix
\newline $x$ is a vector of unknown variables, $n\times 1$ column vector
\newline $b$ is $m \times 1$ column vector

Lipschutz says that a system of linear equations is a list of linear equations with same unknowns (2009,pp.58). In this problem statement we have 3 linear equations with 3 unknowns in a $3\times 3$ system.

The vector addition and scaler multiplication properties of a vector are given by,

\begin{equation}
    If\ a= \begin{bmatrix}
        a_1\\a_2\\.\\.\\a_n
    \end{bmatrix},b= \begin{bmatrix}
        b_1\\b_2\\.\\.\\b_n
    \end{bmatrix}, then\ a+b=\begin{bmatrix}
        a_1+b_1\\a_2+b_2\\.\\.\\a_n+b_n
    \end{bmatrix}
\end{equation}

\begin{equation}
    If\ 'k' \ is\ a\ scaler,\ a = \begin{bmatrix}
        a_1\\a_2\\.\\.\\a_n
    \end{bmatrix}, then\ ka=\begin{bmatrix}
        ka_1\\ka_2\\.\\.\\ka_n
    \end{bmatrix}
\end{equation}

Referring to these properties, we can separate and extract 3 vectors from the given system of linear equations:

\[\
u:x\begin{bmatrix}
    1 \\
    2 \\
    5
\end{bmatrix} , v:y\begin{bmatrix}
    1 \\
    3 \\
    6
\end{bmatrix} , w:x\begin{bmatrix}
    -1 \\
    -8 \\
    -10
\end{bmatrix}\]

We can further rewrite the system of equations in below form:

\begin{align}
\begin{bmatrix}
    1&1&-1 \\
    2&3&-8 \\
    5&6&-10
\end{bmatrix} \begin{bmatrix}
    x \\
    y \\
    z
\end{bmatrix} = \begin{bmatrix}
    -3 \\
    -18 \\
    -25
\end{bmatrix} 
\end{align}

Comparing (15) to formula (A), we have the representation of the system of linear equations in the Matrix vector equation.

                                                                        
\subsubsection*{b) Calculate determinant}

Determinant is an essential tool for gathering and inspecting properties of square matrices (Lipschutz,2009). In this problem statement, the given system of lienar equations has a unique solution if and only if its determianat i.e. $|A| \neq 0$.

\[\det A= \begin{vmatrix}
    1&1&-1 \\
    2&3&-8 \\
    5&6&-10
\end{vmatrix} \]
    
\begin{align*}
    |A| &= (1)[(3)\times(-10) - (-8)\times6]-(1)[(2)\times(-10)-(-8)\times(5)]+(-1)[2\times6-3\times5] \\
    &=18-20+3
\end{align*}

\begin{align}
    \boxed{|A|= 1}
\end{align}

Since $|A| \neq 0$ , this system has a unique solution. 

\subsubsection*{c) Solving this system of linear equations}

The linear equations can be presented as an Augmented matrix as below:


$$A=\left[\begin{array}{ccc|c}
1&1&-1& -3 \\
    2&3&-8&-18 \\
    5&6&-10&-25
\end{array}\right]$$
\vspace{2mm}

Using Gaussian elimination, we find the Row Echelon Form of the matrix as below:

\vspace{2mm}

\vspace{2mm}

$\left[\begin{array}{ccc|c}
1&1&-1& -3 \\
    2&3&-8&-18 \\
    5&6&-10&-25
\end{array}\right] \rightarrow\ R_2=R_2-2R_1\rightarrow\left [\begin{array}{ccc|c}
1&1&-1& -3 \\
    1&1&-6&-12 \\
    5&6&-10&-25
\end{array}\right] \rightarrow\ R_3=R_3-5R_1\rightarrow\left [\begin{array}{ccc|c}
1&1&-1& -3 \\
    1&1&-6&-12 \\
    0&1&-5&-10
\end{array}\right] \rightarrow\ R_3=R_3-R_2\rightarrow\left [\begin{array}{ccc|c}
1&1&-1& -3 \\
    1&1&-6&-12 \\
    0&0&1&2
\end{array} \right] \rightarrow\ R_1=R_1-R_2\rightarrow\left [\begin{array}{ccc|c}
1&0&5& 9 \\
    0&1&-6&-12 \\
    0&0&1&2
\end{array}\right] \rightarrow R_2=R_2+6R_3\rightarrow \left [\begin{array}{ccc|c}
1&0&5& 9 \\
    0&1&0&0 \\
    0&0&1&2
\end{array}\right] \rightarrow R_1=R_1-5R_3\rightarrow \left [\begin{array}{ccc|c}
1&0&0& -1 \\
    0&1&0&0 \\
    0&0&1&2
\end{array}\right]$

\vspace{10mm}

From the reduced Echelon form matrix, we see that 
\[\ \boxed{
\begin{bmatrix}
    x \\
    y \\
    z
\end{bmatrix} = \begin{bmatrix}
    -1 \\
    0 \\
    2
\end{bmatrix} }\]

In other words, $x=-1, y=0, z=2$

\vspace{10mm}

\section*{Question 4}

For a linear transformation T: $\mathbf{R}^3 -> \mathbf{R}^4$ . 
\[\ T= \begin{bmatrix}
    x_1\\
    x_2\\
    x_3
\end{bmatrix} = \begin{bmatrix}
    2x_1+3x_2+x_3 \\
    x_1-3x_2 \\
    x_1+x_2+x_3 \\
    3x_1+2x_2+x_3
\end{bmatrix}\]

a) Determine the transformation matrix A 

b) Determine rank(A) 

c) Find the kernel and image of transformation T, and their dimension 

\textit{Provide appropriate justification and explanation to all your answers, detailing the methods used.}

\subsection*{Solution:}

\subsubsection*{a) Transformation matrix A}


We have the property of linear tranformation:

\begin{align*}
\begin{split}
        T(\alpha_1v_1,\alpha_2v_2,..,\alpha_nv_n) = \alpha_1T(v_1)+\alpha_2T(v_2)+...+\alpha_nT(v_n)
\end{split}\\
\begin{split}
    \ for\ all\ v_1,v_2...v_n\in V\ and\ \alpha_1,\alpha_2...\alpha_n\ are\ scalers
\end{split}
\end{align*}

Applying the same to given $T(x)$, we get

\[\ T(x)  = \begin{bmatrix}
    2x_1+3x_2+x_3 \\
    x_1-3x_2 \\
    x_1+x_2+x_3 \\
    3x_1+2x_2+x_3
\end{bmatrix} = x_1\begin{bmatrix}
    2 \\
    1 \\
    1 \\
    3 
\end{bmatrix} +x_2 \begin{bmatrix}
    3 \\-3 \\ 1\\2
\end{bmatrix} +x_3 \begin{bmatrix}
    1\\0\\1\\1
\end{bmatrix} \]

\begin{align}
    T(x)=\begin{bmatrix}
    2&3&1 \\
    1&-3&0 \\
    1&1&1 \\
    3&2&1
\end{bmatrix} \begin{bmatrix}
    x_1\\x_2\\x_3
\end{bmatrix} 
\end{align}

\vspace{5mm}
We have a Linear transformation formula:

\begin{equation}
    \color{blue} T(x) = Ax \tag{A}
\end{equation}

Comparing equation (17) with formula (A), we get matrix $A$:

\begin{align}  A = \begin{bmatrix}
    2&3&1 \\
    1&-3&0 \\
    1&1&1 \\
    3&2&1
\end{bmatrix} \end{align}

\vspace{3mm}

\subsubsection*{b) Determining the $rank(A)$}

In order to find solutions of a linear system, there is a method to eliminate variables step by step using a technique known as Gaussian Elimination (Lipschutz,2009). With this reduction, we get a system of equations in its matrix form callen Row Echelon Form (REF). The below transitions of matrices demonstrate the Gaussian elimination to get the matrrix $A$ in its REF.

\newpage

\vspace{10mm}

$\left[\begin{array} {ccc}
    2&3&1 \\
    1&-3&0 \\
    1&1&1 \\
    3&2&1
\end{array} \right] \rightarrow R_1 = \frac{R_1}{2} \rightarrow \ \left[ \begin{array} {ccc}
    1&\frac{3}{2}&\frac{1}{2} \\
    1&-3&0 \\
    1&1&1 \\
    3&2&1
\end{array} \right] \rightarrow  R_2=R_3-R_2 \rightarrow \left [\begin{array} {ccc}
    1&\frac{3}{2}&\frac{1}{2} \\
    0&4&1 \\
    1&1&1 \\
    3&2&1
\end{array} \right]\rightarrow  R_3=3R_3-R_4 \rightarrow \left[\begin{array} {ccc}
    1&\frac{3}{2}&\frac{1}{2} \\
    0&4&1 \\
    0&1&2 \\
    3&2&1
\end{array} \right] \rightarrow  R_4=R_4-3R_1 \rightarrow  \left[\begin{array}{ccc}
    1&\frac{3}{2}&\frac{1}{2} \\
    0&4&1 \\
    0&1&2 \\
    0&\frac{-5}{2}&\frac{-1}{2}
\end{array}\right] \rightarrow R_2=\frac{R_2}{4} \rightarrow \left[\begin{array}{ccc}
    1&\frac{3}{2}&\frac{1}{2} \\
    0&1&\frac{1}{4} \\
    0&1&2 \\
    0&\frac{-5}{2}&\frac{-1}{2}
\end{array}\right] \rightarrow R_4=\frac{5}{2}R_3+R_4 \rightarrow \left[\begin{array}{ccc}
    1&\frac{3}{2}&\frac{1}{2} \\
    0&1&\frac{1}{4} \\
    0&1&2 \\
    0&0&\frac{9}{2}
\end{array}\right] \rightarrow R_3=R_3-R_2 \rightarrow \left[\begin{array}{ccc}
    1&\frac{3}{2}&\frac{1}{2} \\
    0&1&\frac{1}{4} \\
    0&0&\frac{7}{4} \\
    0&0&\frac{9}{2}
\end{array}\right] \rightarrow R_3=\frac{4}{7} R_3 \rightarrow \left[\begin{array}{ccc}
    1&\frac{3}{2}&\frac{1}{2} \\
    0&1&\frac{1}{4} \\
    0&0&1 \\
    0&0&\frac{9}{2}
\end{array}\right] \rightarrow R_4=\frac{9}{2} R_3 - R_4 \rightarrow \left[\begin{array}{ccc}
    1&\frac{3}{2}&\frac{1}{2} \\
    0&1&\frac{1}{4} \\
    0&0&1 \\
    0&0&0
\end{array}\right] \rightarrow R_2=R_2-\frac{1}{4} R_3 \rightarrow \left[\begin{array}{ccc}
    1&\frac{3}{2}&\frac{1}{2} \\
    0&1&0\\
    0&0&1 \\
    0&0&0
\end{array}\right] \rightarrow R_1=R_1-\frac{1}{2} R_3 \rightarrow \left[\begin{array}{ccc}
    1&\frac{3}{2}&0 \\
    0&1&0\\
    0&0&1 \\
    0&0&0
\end{array}\right] \rightarrow R_1=R_1-\frac{3}{2} R_2 \rightarrow \left[\begin{array}{ccc}
    1&0&0 \\
    0&1&0\\
    0&0&1 \\
    0&0&0
\end{array}\right] $

\vspace{5mm}

In the Row Echelon form, the number of pivot columns or linearly independent columns is 3.

\begin{equation*}
    \boxed{ rank(A) = 3}
\end{equation*}

\newpage

\subsection*{c) Kernel and Image of Transformation and theor dimension}

\paragraph{c.1) Determining Kernel $ker(T)$}

\subparagraph{}

As per definition, if $F:V->U$, then kernel of $F$ is the set of elements in $V$ that map into the zero vector $0$ in $U$ (Lipschutz,2009),

\begin{equation}
    \color{blue} Ker\ F=\{v\in V: F(v)=0\} \tag{B}
\end{equation}

So essentially, kernel is a set of all inputs taken to zero. Hence, for the given linear transformation, we have,

\begin{align*}
    ker(T)= \{x\in R^3 | Ax=0\}
\end{align*}

Using the matrix in Row Echelon Form (REF), and rewriting it in in form $Ax=0$  as below:

\[ \begin{bmatrix}
    1&0&0 \\
    0&1&0\\
    0&0&1 \\
    0&0&0
\end{bmatrix} \cdot \begin{bmatrix}
    x_1 \\
    x_2 \\
    x_3
\end{bmatrix} = \begin{bmatrix}
    0 \\
    0 \\
    0
\end{bmatrix}\]

\vspace{5mm}
Thus, the kernel contains only zero vector with a dimension 3: \[ \boxed{ ker(T)=\begin{bmatrix}
    0 \\
    0 \\
    0
\end{bmatrix}} \]

\paragraph{c.2) Determining Image} 

\subparagraph{}

The $image$ is a span of column vectors of matrix $A$ (UoL,2025). It is given by,

\begin{align*} \tag{\color{blue}C}
    \color{blue} Im(T)&\color{blue}=Ax|x\in R^n\\
    &\color{blue}=(\alpha_1,\alpha_2...\alpha_n)x|x\in R^n \\
    &=\color{blue} span[(\alpha_1,\alpha_2...\alpha_n)] \in R^n
\end{align*}


Using this formula, we compute $Im(T)$ as the column space of REF of matrix $A$ as below:

\begin{align*}
\boxed{
    Im(T)=span[\ \begin{pmatrix} \begin{bmatrix}
        2\\1\\1\\3
    \end{bmatrix},\begin{bmatrix}
        3\\-3\\1\\2
    \end{bmatrix}, \begin{bmatrix}
        1\\0\\1\\1
    \end{bmatrix}\end{pmatrix}  ]}
\end{align*}

\vspace{10mm}

\section*{Question 5}

 \[Matrix ,A=\begin{bmatrix}
    2&-3\\1&6
\end{bmatrix}\]

a) Determine the inverse of matrix A 

b) Find the singular value decomposition (SVD) of matrix A 

c) Given the SVD of matrix, find the SVD of  

\textit{Provide appropriate justification and explanation to all your answers, detailing the methods used.}

\subsection*{Solution:}

\subsubsection*{a) Determine Inverse of $A$}

\[A=\begin{bmatrix}
    2&-3\\1&6
\end{bmatrix}\]

\vspace{3mm}
A square matrix $A \in R^{n\times n}$ is said to be invertible if its deteminant is not equal to 0 (Lipschutz,2009).

\begin{align}
    |A| &= 2 \times 6 - (-3) \times 1 = 15
\end{align}


Since $|A| \neq 0$, the given matrix $A$ is invertible.


We have the below mathematical formula to determine inverse of a matrix:

\begin{equation}
    \color{blue} A^{-1}=\frac{1}{|A|}\cdot \left [\begin{array}{cc}
    a&b\\c&d
\end{array}\right]^{-1} = \frac{1}{|A|}\cdot \left [\begin{array}{cc}
    d&-b\\-c&a
\end{array}\right] \tag{B}
\end{equation}

Applying formula (B) and using deteminant from equation (19), we get,

\[ A^{-1}=\frac{1}{15} \times \begin{bmatrix}
    6&3\\-1&2
\end{bmatrix} =\begin{bmatrix}
    6\cdot \frac{1}{15}&(3)\cdot \frac{1}{15}\\-1\cdot \frac{1}{15}&2 \cdot \frac{1}{15}
\end{bmatrix} \]


\[  \boxed{A^{-1}= \begin{bmatrix}
    \frac{2}{5}&\frac{1}{5}\\ \frac{-1}{15}&\frac{2}{15}
\end{bmatrix}}\]



\subsubsection*{b) Singular Value Decomposition (SVD) Calculation of matrix $A$}

Singular Value Decomposition states that any matrix $A$ could be decomposed into a product of three matrices - two orthogonal matrices $U$ and $V$, and a diagonal matrix $\Sigma$.


The SVD of a matrix $A$ is given by a decomposition of the below form:

\begin{equation}
    \color{blue} A=U\cdot \Sigma \cdot V^T \tag{C}
\end{equation}

\newpage
Denoted as:

\begin{equation}
    \underbracket{A}_{original\ matrix}=\underbracket{U}_{Orthogonal\ matrix,\ L(A)} \cdot \underbracket{\Sigma}_{Diagonal\ matrix} \cdot \underbracket{ V^T}_{orthogonal\ matrix,\ R(A)} 
\end{equation}


\vspace{5mm}
Since the given matrix $A$ is $2\times 2$ , the dimensions of SVD matrices will be,

\begin{align*}
    A_{n\times n}=U_{n\times n}\Sigma _{n\times n} V^T_{n\times n}
\end{align*}



\paragraph{Step 1: Find the right singular vectors of $A$ to compute matrix $V^T$}

\subparagraph{}

We will perform below sub-tasks in this step:
\newline i) Compute $A^TA$. 
\newline ii) Find Eigenvalues of $A^TA$.
\newline iii) Using eigenvalues from step ii., compute eigenvectors of $A^TA$
\newline iv) Find $V^T$

\textbf{i) Determine $A^TA$}


\[ A^TA =\begin{bmatrix}
    2&1\\-3&6
\end{bmatrix} \cdot \begin{bmatrix}
    2&-3\\1&6
\end{bmatrix}\]

\[=\begin{bmatrix}
    4+1&-6+6\\-6+6&9+36
\end{bmatrix}\]

\begin{align}
A^TA = \begin{bmatrix}
    5&0\\0&45
\end{bmatrix}
\end{align}


\textbf{ii)Find Eigen values of $A^TA$ }

As per the definition, for a square matrix $A$, a scaler $\lambda$ is called \textit{eigenvalue} of $A$ if there exists a non-zero column vector $v$ such that (Lipschutz,2009),

\begin{align*}
    Av=\lambda v \\
    Av-\lambda v=0 \\ \tag{\color{blue}D}
    \color{blue} (A-\lambda I)v=0
\end{align*}

Any vector satisfying formula (D) is called \textit{eigenvector} of $A$ corresponding to \textit{eigenvalue} $\lambda$. 

 This equation has a non-zero solution if and only if,

\begin{equation}
    \color{blue} |A-\lambda I|=0 \tag{E}
\end{equation}

Using formula (E) to determine eigenvalues as below: 

\begin{align*}
    |A^TA-\lambda I|=0 
\end{align*}


\[\begin{vmatrix}
    \begin{bmatrix}
    5&0\\0&45
\end{bmatrix} - \lambda \begin{bmatrix}
    1&0\\0&1
\end{bmatrix} \end{vmatrix} = 0\]

\[\begin{vmatrix}
    5-\lambda&0\\0&45-\lambda
\end{vmatrix}=0\]

\begin{align*}
    (5-\lambda)(45-\lambda)=0
\end{align*}

Computing eigenvalues as,

\begin{align*}
    \begin{split}
        5-\lambda=0,&\lambda=-5
    \end{split}\\
    \begin{split}
        45-\lambda=0,&\lambda=-45
    \end{split}
\end{align*}

The eigen values are: 

\begin{align}
    \lambda = 5\  and\  \lambda=45
\end{align}


\textbf{iii)Find Eigenventors}

Using eigen values from (22), we now find Eigen vectors by rewriting formula (D) in terms of $x$:


\begin{equation*}
    (A^TA-\lambda I)x=0 
\end{equation*}


\[\begin{pmatrix}
\begin{bmatrix}
    5&0\\0&45
\end{bmatrix}-\lambda \begin{bmatrix}
    1&0\\0&1
\end{bmatrix} \end{pmatrix} \begin{bmatrix}
    x_1\\x_2
\end{bmatrix} =\begin{bmatrix}
    0\\0
\end{bmatrix} \]

 \[ \begin{pmatrix}
\begin{bmatrix}
    5&0\\0&45
\end{bmatrix}- \begin{bmatrix}
    \lambda&0\\0&\lambda
\end{bmatrix} \end{pmatrix} \begin{bmatrix}
    x_1\\x_2
\end{bmatrix} =\begin{bmatrix}
    0\\0
\end{bmatrix} \]

\textit{Calculating for $\lambda=45$}

\[\begin{pmatrix}\begin{bmatrix}
    5&0\\0&45
\end{bmatrix}- \begin{bmatrix}
    45&0\\0&45
\end{bmatrix} \end{pmatrix} \begin{bmatrix}
    x_1\\x_2
\end{bmatrix}=\begin{bmatrix}
    0\\0
\end{bmatrix}\]

\[ \begin{bmatrix}
    -40&0\\0&0
\end{bmatrix} \begin{bmatrix}
    x_1\\x_2
\end{bmatrix}=\begin{bmatrix}
    0\\0
\end{bmatrix}\]

\vspace{5mm}

Applying Gaussian elimination to the augmented matrix to reduce it to Row Echelon Form:

\vspace{5mm}

$\left [ \begin{array}{cc|c}
-40&0&0\\0&0&0
\end{array} \right ] \rightarrow R_1\times-\frac{1}{40} \rightarrow \left [\begin{array}{cc|cc}
     1&0&0  \\
     0&0&0 
\end{array} \right]$



\newpage Solving for null space:

\[ \begin{bmatrix}
    1&0\\0&0
\end{bmatrix} \begin{bmatrix}
    x_1\\x_2
\end{bmatrix}=\begin{bmatrix}
    0\\0
\end{bmatrix}\]

\vspace{5mm}

Solving this system in terms of free variables, we get below values of $x_1$ and $x_2$ :

$x_1=0$ ,
$x_2=x_2$

The solution vector is as given below:

\[\begin{bmatrix}
    x_1\\x_2
\end{bmatrix}=\begin{bmatrix}
    0\\x_2
\end{bmatrix}\]

\vspace{5mm}

Representing it as a linear combination of vectors:

\[\ \begin{bmatrix}
    x_1\\x_2
\end{bmatrix}=x_2\begin{bmatrix}
    0\\1
\end{bmatrix}\]

For $\lambda = 45$ , the eigen vector is $\left [\begin{array}{c}
     0  \\
     1 
\end{array}\right]$

\vspace{8mm}

\textit{Calculating for $\lambda = 5$:}

Reiterating the eigenvector formula:

\begin{equation*}
    (A-\lambda I)x=0 
\end{equation*}

\[\begin{pmatrix}\begin{bmatrix}
    5&0\\0&45
\end{bmatrix}- \begin{bmatrix}
    5&0\\0&5
\end{bmatrix} \end{pmatrix} \begin{bmatrix}
    x_1\\x_2
\end{bmatrix}=\begin{bmatrix}
    0\\0
\end{bmatrix}\]

\[ \begin{bmatrix}
    0&0\\0&40
\end{bmatrix} \begin{bmatrix}
    x_1\\x_2
\end{bmatrix}=\begin{bmatrix}
    0\\0
\end{bmatrix}\]

\vspace{5mm}

Applying Gaussian elimination to the augmented matrix to reduce it to Row Echelon Form. The transformation begins by swapping the rows because as per Row-Echelon matrix property, all rows with only zeros must be at the bottom of the matrix. (Vincent, 2021)

\vspace{5mm}

$\left [ \begin{array}{cc|c}
0&0&0\\0&40&0
\end{array} \right ] \rightarrow Swapping R_2 and R_1\rightarrow\left [ \begin{array}{cc|c}
0&40&0\\0&0&0
\end{array} \right ]\rightarrow R_2\frac{1}{40} \rightarrow \left [\begin{array}{cc|cc}
     0&1&0  \\
     0&0&0 
\end{array} \right]$

\vspace{8mm}

Solving for null space:

\[ \begin{bmatrix}
    0&1\\0&0
\end{bmatrix} \begin{bmatrix}
    x_1\\x_2
\end{bmatrix}=\begin{bmatrix}
    0\\0
\end{bmatrix}\]

\vspace{5mm}

Solving this system in terms of free variables, we get below values of $x_1$ and $x_2$ :

$x_2=0$
$x_1=x_1$ 

Hence, the solution vector is as given below:

\[\begin{bmatrix}
    x_1\\x_2
\end{bmatrix}=\begin{bmatrix}
    x_1\\0
\end{bmatrix}\]
Representing it in a linear combination of vectors:
\[\ \begin{bmatrix}
    x_1\\x_2
\end{bmatrix}=x_1\begin{bmatrix}
    1\\0
\end{bmatrix}\]

For $\lambda = 5$ , the eigen vector is $\left [\begin{array}{c}
     1  \\
     0 
\end{array}\right]$

\vspace{10mm}

\textbf{iv)Find $V^T$}

The right singular vector or orthogonal matrix is a cmbination of eigen vectors of $A^TA$ .

Hence we can write it as below:

\[V=\begin{bmatrix}
    0&1\\1&0
\end{bmatrix}\]

\begin{equation}
 \boxed{ V^T=\begin{bmatrix}
    0&1\\1&0
\end{bmatrix}}
\end{equation}

\paragraph{Step 2: Find $\Sigma$}

\subparagraph{}

$\Sigma$ is an $n\times n$ matrix with singular values of $A$ on the main diagonal and the rest of the entries as 0 (Zuniga,2021).
The singular values $\sigma_i$  of $A$ are square roots of eigen values of $A^TA$ .
Thus, referring to (22), we get

\begin{align}
    \sigma_1=\sqrt{5}\ and\ \sigma_2=\sqrt{45} 
\end{align}

\begin{align}
   \boxed{\Sigma=\begin{bmatrix}
    \sqrt{5}&0\\0&\sqrt{45}
\end{bmatrix} }
\end{align}



\newpage
\paragraph{Step 3: Find left signular vectors of $A$ for matrix $U$}

\subparagraph{}

Let us find the left singular vectors $u_1$ and $u_2$ for matrix $U$

From formula (C), we have, 

$$A=U\cdot\Sigma \cdot V^T$$

Multiply each side by $V$,

$$ A\cdot V=U\cdot\Sigma \cdot V^T\cdot V$$

Since $V$ and $V^T$ are orthogonal, it follows that $V^TV=I$. Therefore the equation we get is,

$$A\cdot V=U\cdot \Sigma$$

Thus, for finding invidual left singular vectors $u_i$ , we can use the below formula,

\begin{align*} \tag{\color{blue}F} 
\begin{split}
    \color{blue} A\cdot v_i=u_i\cdot\sigma_i   
\end{split} \\
\begin{split}
    \color{blue}u_i=\frac{A\cdot v_i}{\sigma_i}
\end{split}
\end{align*}
    

Using formula (F), to find vectors $u_1$ and $u_2$:

\begin{align*}
\begin{split}
    u_1&=\frac{1}{\sigma_1}\cdot A\cdot v_1 
\end{split}\\
\begin{split}
    = \frac{1}{\sqrt{5}} \cdot\begin{bmatrix}
    2&-3\\1&6
\end{bmatrix} \cdot \begin{bmatrix}
    0\\1
\end{bmatrix} = \begin{bmatrix}
    \frac{2}{\sqrt{5}}&\frac{-3}{\sqrt{5}}\\\frac{1}{\sqrt{5}}&\frac{6}{\sqrt{5}}
\end{bmatrix} \cdot \begin{bmatrix}
    0\\1
\end{bmatrix}
\end{split} \\
\begin{split}
        u_1&= \begin{bmatrix}
    \frac{-3}{\sqrt{5}} \\ \frac{6}{\sqrt{5}}
\end{bmatrix}
\end{split}
\end{align*}



\begin{align*}
    \begin{split}
        u_2&=\frac{1}{\sigma_2}\cdot A\cdot v_2
    \end{split} \\
    \begin{split}
        = \frac{1}{\sqrt{45}} \cdot\begin{bmatrix}
    2&-3\\1&6
\end{bmatrix} \cdot \begin{bmatrix}
    1\\0
\end{bmatrix} = \begin{bmatrix}
    \frac{2}{\sqrt{45}}&\frac{-3}{\sqrt{45}}\\\frac{1}{\sqrt{45}}&\frac{6}{\sqrt{45}}
\end{bmatrix} \cdot \begin{bmatrix}
    1\\0
\end{bmatrix}
    \end{split} \\
    \begin{split}
       u_2&= \begin{bmatrix}
    \frac{2}{3\sqrt{5}} \\ \frac{1}{3\sqrt{5}}
\end{bmatrix} 
    \end{split}
\end{align*}

Computing $U$ as a combination of $u_1$ and $u_2$ as,

\[\ U=\begin{bmatrix}
    u_1&u_2
\end{bmatrix} = \begin{bmatrix}
    \frac{-3}{\sqrt{5}}&\frac{2}{3\sqrt{5}}\\\frac{6}{\sqrt{5}}&\frac{1}{3\sqrt{5}}
\end{bmatrix} \]

\begin{align}
    \boxed{U= \frac{1}{\sqrt{5}}\cdot\begin{bmatrix}
    -3&\frac{2}{3}\\6&\frac{1}{3}
\end{bmatrix}}
\end{align} 

\paragraph{Step 4: Calculate SVD of matrix $A$}

\subparagraph{}


From results (23),(25),(26), we calculate the SVD of matrix $A$, which should be exactly equal to the original given matrix $A$.

$$A=U\cdot \Sigma \cdot V^T$$

\[=\frac{1}{\sqrt{5}}\cdot\begin{bmatrix}
    -3&\frac{2}{3}\\6&\frac{1}{3}
\end{bmatrix}\cdot \begin{bmatrix}
    \sqrt{5}&0\\0&\sqrt{45}
\end{bmatrix}\cdot \begin{bmatrix}
    0&1\\1&0
\end{bmatrix}\]

\[=\begin{bmatrix}
    \frac{-3}{\sqrt{5}}&\frac{2}{3\sqrt{5}}\\\frac{6}{\sqrt{5}}&\frac{1}{3\sqrt{5}}
\end{bmatrix} \cdot \begin{bmatrix}
    \sqrt{5}&0\\0&\sqrt{45}
\end{bmatrix}\cdot \begin{bmatrix}
    0&1\\1&0
\end{bmatrix} \]

Simplifying the denominator for matrix $U$ by multiplying all elements by $\frac{\sqrt{5}}{\sqrt{5}}$ and rewriting $\sqrt{45}$ as $3\sqrt{5}$

\begin{align*}
    \begin{split}
        A &= \begin{bmatrix}
    \frac{-3\sqrt{5}}{5}&\frac{2\sqrt{5}}{15}\\\frac{6\sqrt{5}}{5}&\frac{\sqrt{5}}{15}
\end{bmatrix} \cdot \begin{bmatrix}
    \sqrt{5}&0\\0&3\sqrt{5}
\end{bmatrix}\cdot \begin{bmatrix}
    0&1\\1&0
\end{bmatrix}
    \end{split} \\
    \begin{split}
        &=\begin{bmatrix}
    \frac{-3 \times 5}{5}+0&0+\frac{2\times 3\times 5 }{15}\\\frac{6\times 5}{5}+0&0+\frac{5\times 3}{15}
\end{bmatrix} \cdot \begin{bmatrix}
    0&1\\1&0
\end{bmatrix}
    \end{split} \\
    \begin{split}
        &=\begin{bmatrix}
    -3&2\\6&1
\end{bmatrix} \cdot \begin{bmatrix}
    0&1\\1&0
\end{bmatrix}
    \end{split}
\end{align*}

\begin{equation}
    \boxed{A =\begin{bmatrix}
     2&-3\\1&6 \end{bmatrix}}
\end{equation}

And this is our original given matrix $A$.


\subsubsection*{c) Determine SVD of $A^T$}

For the given matrix $A$, its tranpose is computed as, \[A^T=\begin{bmatrix}
    2&1\\-3&6
\end{bmatrix}\]

Referring to SVD formula (C), its transpose version as per the property of transpose is given by,

\begin{align*}
    A^T&=(U\cdot \Sigma \cdot V^T)^{T} \\
    &=U^T\cdot \Sigma^T \cdot V
\end{align*}

Since the singular values in $\Sigma$ remain the same, we reorder the terms (Reilly,2025) in the above equation to derive the SVD of $A^T$,

\begin{equation}
    \color{blue} A^T=V\cdot\Sigma \cdot U^T \tag{G}
\end{equation}


From equations (23), (25) and (26), we compute transpose of $U$, $\Sigma$. We get:


\[U^T=\frac{1}{\sqrt{5}}\cdot\begin{bmatrix}
    -3&6\\ \frac{2}{3}&\frac{1}{3}
\end{bmatrix} , \Sigma^T=\begin{bmatrix}
    \sqrt{5}&0\\0&\sqrt{45}
\end{bmatrix}, V=\begin{bmatrix}
    0&1\\1&0
\end{bmatrix}\]

Calculating SVD of $A^T$ as below:

\begin{align*}
    \begin{split}
        A^T&= \begin{bmatrix}
    0&1\\1&0
\end{bmatrix}\cdot\begin{bmatrix}
    \sqrt{5}&0\\0&\sqrt{45}
\end{bmatrix} \cdot \frac{1}{\sqrt{5}}\cdot\begin{bmatrix}
    -3&6\\ \frac{2}{3}&\frac{1}{3}
\end{bmatrix} 
    \end{split} \\
    \begin{split}
        &=\begin{bmatrix}
    0&1\\1&0
\end{bmatrix}\cdot\begin{bmatrix}
    \sqrt{5}&0\\0&\sqrt{45}
\end{bmatrix} \cdot\begin{bmatrix}
    \frac{-3}{\sqrt{5}}&\frac{6}{\sqrt{5}}\\ \frac{2}{3\sqrt{5}}&\frac{1}{3\sqrt{5}}
\end{bmatrix}
    \end{split} \\
    \begin{split}
        &=\begin{bmatrix}
    0\cdot \sqrt{5}+1\cdot 0&0\cdot 0+1\cdot \sqrt{45}\\
    1\cdot \sqrt{5}+0&0+0
\end{bmatrix} \cdot\begin{bmatrix}
    \frac{-3}{\sqrt{5}}&\frac{6}{\sqrt{5}}\\ \frac{2}{3\sqrt{5}}&\frac{1}{3\sqrt{5}}
\end{bmatrix} 
    \end{split}
\end{align*}



Simplifying the denominator for matrix $U$ by multiplying all elements by $\frac{\sqrt{5}}{\sqrt{5}}$ and rewriting $\sqrt{45}$ as $3\sqrt{5}$

\begin{align*}
    \begin{split}
        A&=\begin{bmatrix}
    0&\sqrt{45}\\ \sqrt{5}&0
\end{bmatrix}\cdot\begin{bmatrix}
    \frac{-3}{\sqrt{5}}&\frac{6}{\sqrt{5}}\\ \frac{2}{3\sqrt{5}}&\frac{1}{3\sqrt{5}}
\end{bmatrix}
    \end{split} \\
    \begin{split}
        &=\begin{bmatrix}
    0&3\sqrt{5}\\ \sqrt{5}&0
\end{bmatrix}\cdot\begin{bmatrix}
    \frac{-3\sqrt{5}}{5}&\frac{6\sqrt{5}}{5}\\ \frac{2\sqrt{5}}{15}&\frac{\sqrt{5}}{15}
\end{bmatrix}
    \end{split}
\end{align*}

\begin{equation}
    \boxed{ A^T=\begin{bmatrix}
    2&1\\-3&6
\end{bmatrix}}
\end{equation}


\newpage
\textbf{References}:

Hass, Joel, et al. Thomas' Calculus in SI Units, Pearson Education, Limited, 2019. ProQuest Ebook Central, http://ebookcentral.proquest.com/lib/liverpool/detail.action?docID=5735657.

Lipschutz, S. (2009) Linear algebra /. New York : McGraw-Hill,.A square matrix $A \in R^{n\times n}$ is said to be invertible if its deteminant is not equal to 0 (Lipschutz,2008,section 2.9, Inverse of a 2X2 Matrix).

Vincent, R, J. (2021).’Singular Value Decomposition (SVD) — Working Example’. \newline Available: https://medium.com/intuition/singular-value-decomposition-svd-working-example-c2b6135673b5. (Accessed: 05 May 2025)


Zuniga, C. and Society of Photo-optical Instrumentation Engineers, publisher (2021) \textit{Singular value decomposition for imaging applications / Christian Zuniga.} Bellingham, WA: Society of Photo-Optical Instrumentation Engineers. Available at: https://doi.org/10.1117/3.2611523.

Reilly, J. (2025) \textit{Fundamentals of Linear Algebra for Signal Processing / by James Reilly.} 1st ed. 2025. Cham: Springer Nature Switzerland. Available at: https://doi.org/10.1007/978-3-031-68915-4.

Sharma, S. (2022).'Hessian Matrix'. \newline Available:https://sid-sharma1990.medium.com/hessian-matrix-f9863f934075#:~:text=The%20Hessian%20matrix%20will%20always,3×3%20dimension%20matrix. (Accessed: 05 May 2025)

Fred E. Szabo (2015) ‘M’, in The Linear Algebra Survival Guide. Elsevier Inc, pp. 219–233. Available at: https://doi.org/10.1016/B978-0-12-409520-5.50020-5.


\end{document}
